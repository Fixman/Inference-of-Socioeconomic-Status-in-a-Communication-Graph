\subsection{Banking Information}

The dataset used in this study also contains information about account from a big bank in Mexico for a period of 6 months, denoted \( \mathlarger{B} \). Each bank dataset \( b \in \mathlarger{B} \) contains one phone number \( b_p \), which is anonymized with the same hash function user in \( \mathlarger{P} \), and the real income of this person in the 6 months \( b_{s_0}, \ldots, b_{s_5} \). We average these 6 values to get \( b_s \), a measurement of a user's income.

The bank also provides self-reported personal data about the users for a subset \( A \) of the dataset \( B \). For each \( a \in A \), we have \( a_a \), the age of the user, and \( a_g \), its gender. We use this data to manage biases in the inference algorithm, and to measure homophily with user data other than income. We can later use the Reaction Diffusion algorithm \cite{brea2014} to extend the age and gender inference to all of \( B \).

\subsection{Bank and Telco Matching}

Thanks to \( b_p \) being anonimized in the same way that \( p_o \) and \( p_d \), we can match bank users to their mobile phones, and add income data to \( G \) as \( g_s \). The subset of calls \( H \subset G \) where both users have income information from the bank has \num{2027554} users, and \num{5044976} edges which represent \num{29599762} calls and \num{5476783} SMS.
