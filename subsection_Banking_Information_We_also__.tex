\subsection{Banking Information}

We also use information about deposits, withdrawals, and other transactions made by users from one of the largest banks in Mexico. Some of those have data about their phone numbers which is anonymized in the same way as CDRs, so this data can be correlated with the phone data. We use the data from the same \( M \) month period, denoted by \( B \), which include for each user up to 4 phones \( t_0, \cdots, t_3 \), and information about the dollar amount of its transactions \( s_0, \cdots, s_n \).

Just like the previous dataset, \( B \) contains a great amount of data about users that don't belong to the telco. We define \( B_N \) as the bank users which belong to the telco: \( \left( \forall \left( t_0, \cdots, t_3 \right) \in \text{phones}\left( B_N \right) \right) \left( \exists i \right) t_i \in N \).

\subsection{Telco Ground Truth}

For this study we have demographic information about age and gender \( \left< e, g \right> \) from a subset of nodes \( G_{GT} \subset G_N \) which we call \textbf{Ground Truth}. These data is provided by the telco and we don't have any control on the selection process.

To support the study, we got results from the lastest mexican census \cite{mexico2010census} and, using the socioeconomic level of certain areas of Mexico, we can make a first approximation of the wealth in these areas.

\subsection{Demographic Inference}

Given the set of gender and age data in the Ground Truth \( \left| G_{GT} \right| = 1805534 \), and the set of calls made by the users, we can use the Reaction Diffusion algorithm \cite{brea2014} to extend the inference to all of \( G_N \).

