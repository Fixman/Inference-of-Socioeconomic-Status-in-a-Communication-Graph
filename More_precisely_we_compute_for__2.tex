More precisely, we compute for each user $g_k$ its call distribution across the defined income categories and denote it as $\vec{\alpha} =(\alpha_i); i \in [1,5], i\in \mathbb{N}$ . Here $\alpha_i$ was the fraction of outgoing callas records ($p_k$) that user $g_k$ has to the income category $c_i$. These calls distributions $\vec{\alpha_k}$ were used as the parametrization vector of a Dirichlet distribution $D(\vec{\alpha_k})$ for each user $k$. Then, interpreting the sample space of $D(\vec{\alpha_k})$ as a discrete probability distribution and considering the homophily assumption,  we expected that users belonging to $c_i$ tends to be a concentrated to this particular category. In other words we take the Dirichlet sample space $vec{p_1,...p_i}, i\in 1,...5, \sum_i{p_i}=1$ as a proxy of belonging to different income categories.\\