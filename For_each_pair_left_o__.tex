For each pair \( \left< o, d \right> \in {G_N}_L \) we define \( X \), as the set of incomes for callees and \( Y \) as the set of incomes for callers. According to hour hypothesis, \( X \) and \( Y \) should have a significant correlation.

We can use the \textbf{Spearman's rank correlation coefficient} to test the statistical dependence of the sets \( X \) and \( Y \). This is an improvement over the usual Pearson correlation coefficient in that we compare the ranks of each datapoint, and therefore we get a good coefficient on both people with low and high income levels.

\[
r_s = \rho_{\rank(X) \rank(Y)} = \frac{\cov(\rank(x), \rank(y))}{\sigma_{\rank(x)} \sigma_{\rank(y)}}
\]