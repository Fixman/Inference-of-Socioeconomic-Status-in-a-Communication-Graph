\section{Inference Methodology}

%For the SEI:

%Census data + geolocalization + ARPU + cellphone payments (recharges)  --> SEI estimation for geolocalized users. 

%Second, we extended our predictions taking advantage of graph structure and socioeconomic homophily. To this end, we considered a bayesian  

%\sout{As part of the dataset from the bank \( B \), we have the monthly salaries of most bank users \( B_{S_0} \cdots B_{S_t} \) for a period \( t \) larger than \( M \). We considered the average of \( B_{S_i} \) for a period of \( 6 \) moths to generate \( B_S \)}

%\sout{To infer the monthly salary of the users we take the average of \( B_{S_i} \) for a period of \( 6 \) moths to generate \( B_S \), and we compare them with other users' salaries by using the link correlations in \( G_N \).}

One of the major goals in this work is to estimate the income for users in our graph without information from the bank (i.e. for each user $g_k \in G $ $\ g_k \notin g_s$). To this end, we take advantage of both, the income information contained in our bank clients ($g_s$) and the user´s relations given for the calls records. \\

First of all we highlight an homophily trend regarding to income status, i.e. that \textit{telco} user´s with a given socioeconomic level trends to communicate with others of the same status (Figure XXX, Spearman´s Rank correlation $\rho= 0.474$, p-val$<10^{-6}$). 

Then we take advantage of the observed homophily to propagate income information contained in our seeds $g_s$ and address a statistical inference of income status for the remaining graph. \\

Then we define ad-hock 5 income categories according to our use case interest and relevance. These categories were split in ranges $c_1=[1K,2.5K]$; $c_2=[2.5K,7.5K]$; $c_3=[7.5K,20K]$;$c_4=[20K,50K]$; $c_5=[50K,+ \infty]$.  In this way, our goal was to infer an income category $c_i$ for each user $g_k$ having at lest one directed graph link to users in $g_s$ (given for $CDR$ records  $p\in P$).\\

More precisely, we first compute for each user $g_k$ its call distribution  across the defined income categories and denote it as $\vec{\alpha} =(\alpha_i); i \in [1,5], i\in \mathbb{N}$ . Here $\alpha_i$ was the fraction of outgoing callas records $p_k$ that user $g_k$ has to the income category $c_i$. These calls distributions $\vec{\alpha_k}$ were used as  

%To this end, we first have checked a underlaying the hypothesis of homophily, i.e that users of high socioeconomic level trends to relate with others of similar status.

