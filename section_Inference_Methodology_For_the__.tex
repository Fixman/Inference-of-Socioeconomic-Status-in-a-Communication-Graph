\section{Inference Methodology}

%For the SEI:

%Census data + geolocalization + ARPU + cellphone payments (recharges)  --> SEI estimation for geolocalized users. 

%Second, we extended our predictions taking advantage of graph structure and socioeconomic homophily. To this end, we considered a bayesian  

%\sout{As part of the dataset from the bank \( B \), we have the monthly salaries of most bank users \( B_{S_0} \cdots B_{S_t} \) for a period \( t \) larger than \( M \). We considered the average of \( B_{S_i} \) for a period of \( 6 \) moths to generate \( B_S \)}

%\sout{To infer the monthly salary of the users we take the average of \( B_{S_i} \) for a period of \( 6 \) moths to generate \( B_S \), and we compare them with other users' salaries by using the link correlations in \( G_N \).}

The main contribution of this work is to estimate the income of the telco users for which we lack banking data but have bank clients in their neighborhood of network graph. To show the feasibility of this task we first show the existence of a strong income homophily in the telco graph as is evidenced in Figure x. For each pair \( \left< o, d \right> \in {G_N}_L \) we define \( X \), as the set of incomes for callers and \( Y \) as the set of incomes for callees. According to what we can observe in Figure x, \( X \) and \( Y \) should be significantly correlated. Given the broad non gaussian distribution of the income's values, we choose to use a rank based measure of correlation, namely the Spearman's rank correlation to test the statistical dependence of sets \( X \) and \( Y \). Equation for the Spearman's correlation follows below: 

\[
r_s = \mathlarger{\rho}_{\operatorname{rank}(X) \operatorname{rank}(Y)} = \frac{\operatorname{cov}(\operatorname{rank}(x), \operatorname{rank}(y))}{\sigma_{\operatorname{rank}(X)} \sigma_{\operatorname{rank}(Y)}}
\]

Using this coefficient, we get a correlation of \( \num{0.474} \), whith a P-value of \( P < 10^{-6} \) with respect to the null hypothesis where links between users are selected randomly preserving the total amount of links as in the original graph.



%First of all we highlight an homophily trend regarding to income status, i.e. that \textit{telco} user´s with a given socioeconomic level trends to communicate with others of the same status (Figure XXX, Spearman´s Rank correlation $\rho= 0.474$, p-val$<10^{-6}$). 

Then we take advantage of the observed homophily to propagate income information contained in our seeds $g_s$ and address a statistical inference of income status for the remaining graph. \\

As previous data cleaning, we remove outliers in the income distribution (dropping users above the quantile 0.99) and disregard users with insufficient income information (here we drop users with less than $1000\$$). Then we define 2 balanced income categories according to the median of the resulting income distribution and we called it $c_l$ (low income) and $c_h$ (high income). In this way, our goal was to infer an income category $c_l$ or $c_h$ for each user $g_k$ having at lest one directed graph link to users in $g_s$ (given for $CDR$ records  $p\in P$).\\



%To this end, we first have checked a underlaying the hypothesis of homophily, i.e that users of high socioeconomic level trends to relate with others of similar status.
