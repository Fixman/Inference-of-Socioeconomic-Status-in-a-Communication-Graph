\section{Inference Methodology}

%For the SEI:

%Census data + geolocalization + ARPU + cellphone payments (recharges)  --> SEI estimation for geolocalized users. 

%Second, we extended our predictions taking advantage of graph structure and socioeconomic homophily. To this end, we considered a bayesian  

%\sout{As part of the dataset from the bank \( B \), we have the monthly salaries of most bank users \( B_{S_0} \cdots B_{S_t} \) for a period \( t \) larger than \( M \). We considered the average of \( B_{S_i} \) for a period of \( 6 \) moths to generate \( B_S \)}

%\sout{To infer the monthly salary of the users we take the average of \( B_{S_i} \) for a period of \( 6 \) moths to generate \( B_S \), and we compare them with other users' salaries by using the link correlations in \( G_N \).}

The major goal of this work is to estimate the income for users belonging to the graph and not having income information in the graph \(i.e. $g_i \in G $ and $g_i \notin \ g_s $\). To this end, we take advantage of both, the income information contained in our bank clients \(g_s\) and the user´s relations given for the calls records. 
First of all we highlight the presence of homophily regarding to income status, i.e. that telco user´s with a given socioeconomic level trends to communicate with others of the same status (Figure XXX, Spearman´s Rank correlation $\rho= 0.474$, p-val$<10^{-6}$). 

Then we take advantage of the observed homophily to propagate the income information contained in our seeds $g_s$ and address a statistical inference of income status in our directed graph. 

%To this end, we first have checked a underlaying the hypothesis of homophily, i.e that users of high socioeconomic level trends to relate with others of similar status.

