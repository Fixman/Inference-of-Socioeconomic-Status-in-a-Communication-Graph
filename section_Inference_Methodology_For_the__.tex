\section{Inference Methodology}

%For the SEI:

%Census data + geolocalization + ARPU + cellphone payments (recharges)  --> SEI estimation for geolocalized users. 

%Second, we extended our predictions taking advantage of graph structure and socioeconomic homophily. To this end, we considered a bayesian  

%\sout{As part of the dataset from the bank \( B \), we have the monthly salaries of most bank users \( B_{S_0} \cdots B_{S_t} \) for a period \( t \) larger than \( M \). We considered the average of \( B_{S_i} \) for a period of \( 6 \) moths to generate \( B_S \)}

%\sout{To infer the monthly salary of the users we take the average of \( B_{S_i} \) for a period of \( 6 \) moths to generate \( B_S \), and we compare them with other users' salaries by using the link correlations in \( G_N \).}

The main contribution of this work is to estimate the income of the telco users for which we lack banking data but have bank clients in their neighborhood of netwro graph. To show the feesibility of this task we first show the existence of a strong income homophily in the telco graph.

For each pair \( \left< o, d \right> \in {G_N}_L \) we define \( X \), as the set of incomes for callers and \( Y \) as the set of incomes for callees. According to hour hypothesis, \( X \) and \( Y \) should have a significant correlation.

We can use the \textbf{Spearman's rank correlation coefficient} to test the statistical dependence of the sets \( X \) and \( Y \). This is an improvement over the usual Pearson correlation coefficient in that we compare the ranks of each datapoint, and therefore we get a good coefficient on both people with low and high income levels.

\[
r_s = \mathlarger{\rho}_{\operatorname{rank}(X) \operatorname{rank}(Y)} = \frac{\operatorname{cov}(\operatorname{rank}(x), \operatorname{rank}(y))}{\sigma_{\operatorname{rank}(X)} \sigma_{\operatorname{rank}(Y)}}
\]

Using this coefficient, we get a correlation of \( \num{0.474} \), where the P-value of the null hypothesis, which hypothesizes that the income level between callers and callee isn't correlated is \( P < 10^{-6} \).



%First of all we highlight an homophily trend regarding to income status, i.e. that \textit{telco} user´s with a given socioeconomic level trends to communicate with others of the same status (Figure XXX, Spearman´s Rank correlation $\rho= 0.474$, p-val$<10^{-6}$). 

Then we take advantage of the observed homophily to propagate income information contained in our seeds $g_s$ and address a statistical inference of income status for the remaining graph. \\

Then we define ad-hock 5 income categories according to our use case interest and relevance. These categories were split in ranges $c_1=[1K,2.5K]$; $c_2=[2.5K,7.5K]$; $c_3=[7.5K,20K]$;$c_4=[20K,50K]$; $c_5=[50K,+ \infty]$.  In this way, our goal was to infer an income category $c_i$ for each user $g_k$ having at lest one directed graph link to users in $g_s$ (given for $CDR$ records  $p\in P$).\\



%To this end, we first have checked a underlaying the hypothesis of homophily, i.e that users of high socioeconomic level trends to relate with others of similar status.
