\section{Inference Methodology}

%For the SEI:

%Census data + geolocalization + ARPU + cellphone payments (recharges)  --> SEI estimation for geolocalized users. 

%Second, we extended our predictions taking advantage of graph structure and socioeconomic homophily. To this end, we considered a bayesian  

As part of the dataset from the bank \( B \), we have the monthly salaries of most bank users \( B_{S_0} \cdots B_{S_t} \) for a period \( t \) larger than \( M \). We considered the average of \( B_{S_i} \) for a period of \( 6 \) moths to generate \( B_S \)

To infer the monthly salary of the users we take the average of \( B_{S_i} \) for a period of \( 6 \) moths to generate \( B_S \), and we compare them with other users' salaries by using the link correlations in \( G_N \).

In order to propagate the income prediction to others users in our graph, we address a statistical inference of income categories in our directed graph. To this end, we first have checked a underlaying the hypothesis of homophily, i.e that users of high socioeconomic level trends to relate with others of similar status.

