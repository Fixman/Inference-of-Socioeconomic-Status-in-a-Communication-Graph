\section{Results}

\subsection{Validation Process}

We need to make sure that our inference algorithm correctly predicts users' income, we create a binary classifier between two equally sized groups, \textbf{wealthy} users and \textbf{poor} users. Our hypothesis is that if there's a link connecting two users then both of them will be in the same group.

We can use this prediction method to classify the links between users in $ G $ for which we know the income using this hypothesis into four classes depending on the group of each member of the link:

\begin{tabularx}{\textwidth}{ r c c }
& Predicted Wealthy Group & Predicted Poor Group \\
Condition Positive & \cellcolor{green} True Positive & \cellcolor{red} \makecell{False Negative \\ (Type II Error)} \\ 
Condition Negative & \cellcolor{red} \makecell{False Positive \\ (Type I Error)} & \cellcolor{green} True Positive \\
\end{tabularx}

To validate this, we use a \textbf{Received Operating Characteristic Curve} to plot the \textbf{True Positive Rate} and the \textbf{False Positive Rate} of the system. Since the amount of people in each classification group is mostly equal, a random guess would give a point along the diagonal line.

As an objective measurement, we can calculate the \textbf{Area under the curve} as the change that the classified will rank a randomly chosen instance of a wealthy user higher than a randomly chosen instance of a poor one; for this we define the density functions $ f_1(x) $ as the probabilty function that a user will be classified as wealthy, and $ f_0 $ otherwise.

\begin{align*}
\operatorname{TPR}(T) &= \int^{\infty}_T f_1(x) dx \\
\operatorname{FPR}(T) &= \int^{\infty}_T f_0(x) dx \\
A = P(X_1 > X_0) &= \int^{-\infty}_{\infty} \operatorname{TPR}(T) \operatorname{FPR}(T) dT
\end{align*}

\subsection{Results}


To infer this we compute the call distribution for each user $ \mathlarger{G}_W $ to the wealthy category its call distribution to the high income category and denote it as $\alpha)$ . Here $\alpha$ was the fraction of outgoing callas records ($p_k$) that user $g_k$ has to the income category $c_h$. These calls distributions $\alpha_{(k)}$ were used as the parametrization of a Beta distribution $B(\alpha_k)$ for each user $k$. Then, interpreting the sample space of $B(alpha_k)$ as a discrete probability distribution and considering the homophily assumption,  we expected that users belonging to $c_h$ tends to be a concentrated to this particular category. In other words we take the Beta sample space $vec{1-p_h,p_h}$ as a proxy of belonging to different income categories.

