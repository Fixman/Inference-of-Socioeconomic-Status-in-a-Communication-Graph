\section{Data Source}

\subsection{Mobile Phone Data Source}

The data used in this study consist of \textbf{Call Detail Records}, or CDRs, made on calls and text messages from a Mexican \textit{telco} in a period of 3 months, denoted by \( \mathbb{P} \). Every CDR \( p \in \mathbb{P} \)  contains the anonimized phone numbers of the two participants in the communication \( \left< p_o, p_d \right> \), the call time \( p_t \), and, in the case of CDRs for phonecalls the call duration \( p_s \). A subset of these calls also contain the latitude and longitude of the call's antenna, \( \left< p_y, p_x \right> \).

\( \mathbb{P} \) contains calls and messages where the origin or destination phones are users in the \textit{telco}, but it doesn't include communications between the large amount of users which aren't part of this. If we define \( N \) as the users of the telco, and \( \mathbb{P}_N \subset \mathbb{P} \) as the calls where \( \left( \forall p \in \mathbb{P}_N \right) p_o \in N \wedge p_d \in N \), we can create a communications graph \( \mathbb{G} \) which contains all the users from the telco, and every user has each and every one of their calls in the graph.

With the data provided to us by the telco, \( \mathbb{G} \) contains \( ? \) users who made \( ? \) calls with a total of \( ? \) minutes and sent \( ? \) text messages.
