\section{Introduction}

\subsection{Data source}

The data user for this study consist of \textbf{Call Detail Records}, or CDRs, about calls and text messages 

Los datos usados por este estudio consisten de \textbf{Call Detail Records}, o CDRs, sobre llamados y mensajes de texto provenientes de la red de una compañía telefónica Mexicana, en un periodo de \( M \) meses (\( M = 3 \)), denotado por \( L \). Cada CDR contiene los números de teléfonos anonimizados de los dos participantes de la comunicación \(\left<o, d\right>\), el tiempo del llamado \(t\), y, en el caso de los CDRs de llamados su duración \(s\). Un subconjunto de estos llamados también contienen la latitud y la longitud de la antena usada para la llamada, \(\left<y, x\right>\).

\( L \) contiene llamadas y mensajes cuyo origen o destino son los usuarios de la telco, pero no incluye comunicaciones entre la gran cantidad de usuarios que no son parte de esta. Definiendo \( N \) como los usuarios de la telco y \( L_N \) como las llamadas donde \( o \in N \wedge d \in N \), podemos crear un grafo de comunicaciones \( G_N \) que contenga todas las llamadas de cada uno de sus usuarios.

Además se usa información sobre depósitos, extracciones, y transacciones hechas desde bancarias de uno de los bancos más grandes de México, algunas de las cuales incluyen el número de teléfono anonimizado de la misma manera que en los CDRs, por lo cual se puede correlacionar con estos. Usamos los datos durante el mismo periodo de \( M \) meses, denotado por \( B \), los cuales incluyen, por cada usuario hasta 4 teléfonos \( t_0 \cdots t_3 \), e información del monto de sus transacciones \( s_0 \cdots s_n \).

Al igual que en el dataset anterior, \( B \) contiene una gran cantidad de datos sobre usuarios que no pertenecen a la telco. Definimos \( B_N \) como los usuarios del banco donde \( \left( \exists i \right) t_i \in N \), osea que pertenezcan a la telco.

Para este estudio conseguimos información demográfica sobre la edad y el género \( \left<e, g\right> \) de un subconjunto de los nodos \( G_{GT} \subset G_N \) al que denominamos \textbf{Ground Truth}. Estos datos los provee la compañía telefónica y no tenemos ningún tipo de control sobre como se selecciona.

Para apoyar al estudio, conseguimos los resultados del último censo mexicano, encontrado en {URL del censo acá} y, usando el nivel socioeconómico de ciertas áreas de México, podemos hacer una primera aproximación del nivel socioeconómico de la población de esas áreas.

\subsection{Data Inference}

Given the set of gender and age data in the Ground Truth \( \left| G_{GT} \right| = 1805534 \), and the set of calls made by the users, we can use the Reaction Diffusion algorithm \cite{brea2014} to extend the inference to all of \( G_N \).