\section{Data Source}

\subsection{Mobile Phone Data Source}


The data used in this study consist of \textbf{Call Detail Records}, or CDRs, made on calls and text messages from a Mexican telco in a period of \( M \) months (\( M = 3 \)), denoted by \( L \). Every CDR contains the anonimized phone numbers of the two participants in the communication \(\left<o, d\right>\), the call time \( t \), and, in the case of CDRs for phonecalls the call duration \( s \). A subset of these calls also contain the latitude and longitude of the call's antenna, \( \left<y, x\right> \).

\( L \) contains calls and messages whos origin or destination are users in the telco, but it doesn't include communications between the great amount of users which aren't part of this. Defining \( N \) as the users of the telco, and \( L_N \) as the calls where \( o \in N \wedge d \in N \), we can create a communications graph \( G_N \) that contains all calls between their users.

\subsection{Banking Information}

We also use information about deposits, withdrawals, and other transactions made by users from one of the largest banks in Mexico. Some of those have data about their phone numbers which is anonymized in the same way as CDRs, so this data can be correlated with the phone data. We use the data from the same \( M \) month period, denoted by \( B \), which include for each user up to 4 phones \( t_0, \cdots, t_3 \), and information about the dollar amount of its transactions \( s_0, \cdots, s_n \).

Just like the previous dataset, \( B \) contains a great amount of data about users that don't belong to the telco. We define \( B_N \) as the bank users which belong to the telco: \( \left( \forall \left( t_0, \cdots, t_3 \right) \in \text{phones}\left( B_N \right) \right) \left( \exists i \right) t_i \in N \).

For this study we have demographic information about age and gender \( \left< e, g \right> \) from a subset of nodes \( G_{GT} \subset G_N \) which we call \textbf{Ground Truth}. These data is provided by the telco and we don't have any control on the selection process.

To support the study, we got results from the last mexican census, found in {cite census URL} and, using the socioeconomic level of certain areas of Mexico, we can make a first approximation of the wealth in these areas.

\subsection{Ground Truth for Data Inference}

Given the set of gender and age data in the Ground Truth \( \left| G_{GT} \right| = 1805534 \), and the set of calls made by the users, we can use the Reaction Diffusion algorithm \cite{brea2014} to extend the inference to all of \( G_N \).

