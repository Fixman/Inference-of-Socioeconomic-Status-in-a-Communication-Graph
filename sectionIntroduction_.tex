\section{Data Source}

\subsection{Mobile Phone Data Source}

The data used in this study consist of a set \( \mathlarger{P} \) of \textbf{Call Detail Records} (CDRs), made up of voice calls and text messages from a Mexican telecommunication company (\textit{telco}) for a 3 months period.
Every CDR \( p \in \mathlarger{P} \)  contains the anonimized phone numbers of the caller and callee \( \left< p_o, p_d \right> \), the start time \( p_t \), and the call duration \( p_s \) in the case voice calls. 
For a subset of the voice calls, the latitude and longitude of the antenna being used, \( \left< p_y, p_x \right> \), is also given.

\( \mathlarger{P} \) contains calls and messages where the origin or destination phones are users in the \textit{telco}, but it doesn't include communications between the large amount of users which aren't part of this. If we define \( N \) as the users of the telco, and \( \mathlarger{P}_N \subset \mathlarger{P} \) as the calls where \( \left( \forall p \in \mathlarger{P}_N \right) p_o \in N \wedge p_d \in N \), we can create a communications graph \( \mathlarger{G} \) which contains all the users from the telco, and every user has each and every one of their calls in the graph.

With the data provided to us by the telco, \( \mathlarger{G} \) contains \( ? \) users who made \( ? \) calls with a total of \( ? \) minutes and sent \( ? \) text messages.
