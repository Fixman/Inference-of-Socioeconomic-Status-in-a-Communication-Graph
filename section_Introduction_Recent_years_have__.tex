\section{Introduction}

Recent years have seen the emergence of massive datasets, collecting interactions between millions of individuals. The availability of such datasets open the possibility of studying the social structure of a society, and its temporal dynamics, at the scale of a nation.

In particular, mobile phone datasets provide a very rich view into social interactions and the physical movements of large segments of the population. The calls and messages exchanged between people draw a map of the social fabric, whereas the call locations (recorded through cell tower usages) are a projection of users trajectories in space and time, showing patterns of daily and weekly regularity~\cite{gonzalez2008understanding,ponieman2013human,sarraute2015city}.

Demographic factors play an important role in the constitution and preservation of social links. In particular concerning their age, individuals have a tendency to
establish links with others of similar age. This phenomenon is called homophily~\cite{mcpherson2001birds}, and has been verified in communications graph~\cite{blumenstock2010mobile,sarraute2014} as well as the Facebook graph~\cite{ugander2011anatomy}.

Economic factors are also believed to have a determining role in social network dynamics.
However, there are still very few quantitative analyses (on a large scale) on the interplay between economic status of individuals and their social network.
In~\cite{leo2015socioeconomic}, the authors analyze the correlations between mobile phone data and banking transaction information, revealing the existence of social stratification. They also show the presence of homophily respect to socioeconomic indicators, such as income, purchases and debt.

In this work, we leverage the socioeconomic homophily to generate inferences of socioeconomic status in the communication graph.

