\subsection{Banking Information}

For this study we also used account balances for over 10 million clients of a bank in Mexico for a period of 6 months, denoted \( \mathlarger{B} \). The data for each client \( b \in \mathlarger{B} \) contains his phone number \( b_p \), anonymized with the same hash function used in \( \mathlarger{P} \), and the reported income of this person over 6 months \( b_{s_0}, \ldots, b_{s_5} \). We average these 6 values to get \( b_s \), an estimate of a user's income.

The bank also provided us with demographic information for a set of users \( \mathlarger{A} \subseteq \mathlarger{B} \). For each user \( u \in \mathlarger{A} \) we are given  the age \( u_a \) and the gender \( u_g \) of the user. This allowed us to observe differences in the income distribution according to the age and gender. In another line of work, homophily respect to the age has been observed in \cite{brea2014} and used to generate inferences.
